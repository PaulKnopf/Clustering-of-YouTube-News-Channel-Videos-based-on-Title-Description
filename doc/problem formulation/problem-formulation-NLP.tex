\documentclass[12pt,a4paper]{scrartcl}
\UseRawInputEncoding
\usepackage[ngerman]{babel}
\usepackage{amsmath}
\usepackage{listings}
\usepackage{graphicx}
\usepackage{wrapfig}
\usepackage{ragged2e}
\usepackage{blindtext}
\usepackage{titling}
\lstloadlanguages{SQL}

\title{\textbf{Natural Language Processing} \\[0.5em] \large{\textbf{Problem Formulation - Clustering News Outlet Output} \\[0.5em]} }
\date{}
\predate{}
\postdate{}
\posttitle{\par\end{center}}
\author{Paul \& Zoe \& Max}

\begin{document}
\maketitle
\section{Motivation}
\begin{itemize}
    \item Overview of content on news channels broadcasting in English on YouTube.
    \item Comparing the focus of the topic areas of the different news stations with each other.
\end{itemize}
\section{Goal(s)}
\begin{itemize}
	\item \textbf{Clustering of data by topics} (i.e. topics like \dq Ukraine War\dq{}, \dq Economy\dq{}, \dq Politics\dq{}, ...) \\
	\begin{itemize}
		\item Different methods of clustering
		\begin{itemize}
			\begin{minipage}{0.1cm}
				\begin{align*}
					\left[ \begin{array}{lllll}
					\\
					\\
					\\
					\\
					\\
					\end{array}
					\right.
				\end{align*}
			\end{minipage}
			\begin{minipage}{4.5cm}
			\item training based modells
				\begin{itemize}
					\item active learning
					\item supervised learning
				\end{itemize}
			\end{minipage}
			\begin{minipage}{0.1cm}
				\begin{align*}
					\left. \begin{array}{lllll}
					\\
					\\
					\\
					\\
					\\
					\end{array}
					\right]
				\end{align*}
			\end{minipage}
			\begin{minipage}{7cm}
			If needed, since the data is raw, there is no training data. A training data-set would have to be created. Also, in light of newly emerging topics, this is most likely not the best approach.
			\end{minipage}
			\item none training based modells
		\end{itemize}
	\end{itemize}
	\item[\labelitemi] \textbf{Comparison of clusters within single news station}
	\begin{itemize}
		\item Distance between clusters
		\item Size and popularity of clusters
		\item Cluster development over time
		\item Relation between emotionality of title and view count
		\item Sentiments of news channel on identified topics
		\item ...
	\end{itemize}
	\item[\labelitemi] \textbf{Comparing clusters of different news channels}
	\begin{itemize}
		\item Differences and overlaps of clusters
		\item Comparison of similar clusters (i.e. matching categories)
		\item ...
	\end{itemize}
\end{itemize}
\section{Data}
We have collected the \textbf{metadata of uploaded videos} on the following Youtube news channels (DW-News, CNN-News, BBC-News, Al-Jazeera-English, Fox-News and CCTV-Video-News-Agency) since 25-03-2023. The collected information of the videos has the following attributes.\\
\begin{center}
\begin{tabular}{ |c|c|c|c|c|c|c| } 
 \hline
 Title & Views & Video-Length & Description & Time-Upload & Time-Crawling &  Channel\\ 
 \hline
 ... & ... & ... & ... & ... & ... & ... \\
 \hline
\end{tabular}
\end{center}
The data is retrieved every 15 minutes. In the appendix you can find a part of such a data set.
\section{Models/methods/algorithms}
\section{Tasks of each team member with description}
\begin{itemize}
    \item Data processing (Zoe)
    \item Create model overview (Max/Paul)
    \item Problem formulation (Paul)
    \item Power Point presentation (Zoe)
    \item Model application \& tweaking (all)
\end{itemize}
\section{Your plan and schedule for the project steps}
\begin{itemize}
    \item Data processing -
    \item Model selection -
    \item Model application -
    \item Evaluation -
\end{itemize}
\section{Potential problems}
\begin{itemize}
    \item No sufficiently large data set.
    \item Incomplete video description.
    \item During scraping deleted commas and semicolons.
\end{itemize}
\end{document}
